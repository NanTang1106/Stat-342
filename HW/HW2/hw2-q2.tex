\documentclass[12pt,letterpaper]{article}
\usepackage[utf8]{inputenc}
\usepackage{amsmath,amssymb,fullpage,graphicx}
\usepackage{subfigure}
\usepackage{amssymb}

\begin{document}
\subsection*{Q2-2}

\begin{verbatim}
iteration <- 10000
lambda0 <- 1
sample_size <- 20
test_size <- 0.05
critical_value <- abs(qnorm(test_size/2))
count <- 0

for (i in 1:iteration) {
  sample <- rpois(sample_size, lambda0)
  W <- (sum(sample) - sample_size * lambda0) / sqrt(sum(sample))
  if (abs(W) > critical_value) {
    count <- count + 1
  } 
}

type_one_err <- count / iteration
\end{verbatim}

\begin{verbatim}
 > print(type_one_err)
[1] 0.0509
\end{verbatim}

\noindent Under the condition that $\lambda_0 = 1$, sample size $= 5 $, $X_1, X_2, ... X_{20}$ from $Poisson(\lambda_0)$ and $\alpha = 0.05$, I iterated the Wald test for ten thousand times. \\

\noindent There are 509 cases when test statistic $W = \frac{\sum_{i=1}^{20} x_i - n \lambda_0}{\sqrt[]{\sum_{i=1}^{20} x_i}} $ is greater than $Z_{\frac{\alpha}{2}}$, i.e. rejected the null hypothesis. \\

\noindent The type one error is $P_{H_0}($ reject the null $) = 0.0509$ which is close to the size $\alpha$.


\end{document}